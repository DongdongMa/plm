%\VignetteIndexEntry{Introduction to plm}
\documentclass{article}
\usepackage[T1]{fontenc}
\usepackage{ucs}
\usepackage[utf8]{inputenc}

\title{Introduction to plm}

\author{Yves Croissant}
\usepackage{/usr/share/R/share/texmf/Sweave}
\begin{document}

\maketitle


\section{Introduction}

The aim of package \texttt{plm} is to provide an easy way to estimate
panel models. Some panel models may be estimated with package \texttt{nlme}
(\textit{non--linear mixed effect models}), but not in an intuitive
way for an econometrician.
\texttt{plm} provides methods to read panel data, to estimate a wide
range of models and to make some tests. 
This library is loaded using :

\begin{Schunk}
\begin{Sinput}
> library(plm)
\end{Sinput}
\end{Schunk}

This document illustrates the features  of  \texttt{plm}, using
data available in  package \texttt{Ecdat}. 

\begin{Schunk}
\begin{Sinput}
> library(Ecdat)
\end{Sinput}
\end{Schunk}

These data are used in  \textsc{Baltagi} (2001).

\section{Reading data}

With \texttt{plm}, data are stored in an object of class
\texttt{pdata.frame},  which is a \texttt{data.frame} with additional
attributes describing the structure of the data set.
A \texttt{pdata.frame} may be created from an ordinary \texttt{data.frame}
using the \texttt{pdata.frame} function or from a text file using the
\texttt{pread.table} function.


\subsection{Reading the data from a data.frame}

We illustrate the use of the \texttt{pdata.frame} function with the
\texttt{Produc} data :


\begin{Schunk}
\begin{Sinput}
> data(Produc)
> pdata.frame(Produc, "state", "year", "pprod")
\end{Sinput}
\end{Schunk}

The  \texttt{pdata.frame} function has  4 arguments :

\begin{itemize}
\item the name of the  \texttt{data.frame},
\item \texttt{id} : the individual index,
\item \texttt{time} : the time index,
\item \texttt{name} : the name under which the \texttt{pdata.frame}
  will be stored.
\end{itemize}

Observations are assumed to be sorted by individuals first, and by
period. The third argument is optional, if \texttt{NULL} a new
variable called \texttt{time} is added. The fourth argument is also
optional, if \texttt{NULL} the \texttt{pdata.frame} is stored under
the same name as the \texttt{data.frame}.


\begin{Schunk}
\begin{Sinput}
> data(Hedonic)
> pdata.frame(Hedonic, "townid")
\end{Sinput}
\end{Schunk}

In case of a balanced panel, the \texttt{id} may be the number of
individuals. In this case, two new variables (called \texttt{id} and
\texttt{time}) are added.

\begin{Schunk}
\begin{Sinput}
> data(Wages)
> pdata.frame(Wages, 595)
\end{Sinput}
\end{Schunk}

A description of the data is obtained using the \texttt{summary}
method :

\begin{Schunk}
\begin{Sinput}
> summary(Hedonic)
\end{Sinput}
\begin{Soutput}
________________________________________________________________________________ 
___________________________________ Indexes ____________________________________
Individual index : townid
Time index       : time
________________________________________________________________________________ 
_______________________________ Panel Dimensions _______________________________
Unbalanced Panel
Number of Individuals        :  92
Number of Time Obserbations  :  from 1 to 30
Total Number of Observations :  506
________________________________________________________________________________ 
__________________________ Time/Individual Variation ___________________________
no time variation   :  zn indus rad tax ptratio 
________________________________________________________________________________ 
____________________________ Descriptive Statistics ____________________________
       mv              crim                zn             indus        chas    
 Min.   : 8.517   Min.   : 0.00632   Min.   :  0.00   Min.   : 0.46   no :471  
 1st Qu.: 9.742   1st Qu.: 0.08205   1st Qu.:  0.00   1st Qu.: 5.19   yes: 35  
 Median : 9.962   Median : 0.25651   Median :  0.00   Median : 9.69            
 Mean   : 9.942   Mean   : 3.61352   Mean   : 11.36   Mean   :11.14            
 3rd Qu.:10.127   3rd Qu.: 3.67708   3rd Qu.: 12.50   3rd Qu.:18.10            
 Max.   :10.820   Max.   :88.97620   Max.   :100.00   Max.   :27.74            
                                                                               
      nox              rm             age              dis        
 Min.   :14.82   Min.   :12.68   Min.   :  2.90   Min.   :0.1219  
 1st Qu.:20.16   1st Qu.:34.64   1st Qu.: 45.02   1st Qu.:0.7420  
 Median :28.94   Median :38.55   Median : 77.50   Median :1.1655  
 Mean   :32.11   Mean   :39.99   Mean   : 68.57   Mean   :1.1880  
 3rd Qu.:38.94   3rd Qu.:43.87   3rd Qu.: 94.07   3rd Qu.:1.6464  
 Max.   :75.86   Max.   :77.09   Max.   :100.00   Max.   :2.4954  
                                                                  
      rad             tax           ptratio          blacks       
 Min.   :0.000   Min.   :187.0   Min.   :12.60   Min.   :0.00032  
 1st Qu.:1.386   1st Qu.:279.0   1st Qu.:17.40   1st Qu.:0.37538  
 Median :1.609   Median :330.0   Median :19.05   Median :0.39144  
 Mean   :1.868   Mean   :408.2   Mean   :18.46   Mean   :0.35667  
 3rd Qu.:3.178   3rd Qu.:666.0   3rd Qu.:20.20   3rd Qu.:0.39623  
 Max.   :3.178   Max.   :711.0   Max.   :22.00   Max.   :0.39690  
                                                                  
     lstat             townid         time    
 Min.   :-4.0582   29     : 30   1      : 92  
 1st Qu.:-2.6659   84     : 23   2      : 75  
 Median :-2.1747   5      : 22   3      : 60  
 Mean   :-2.2342   83     : 19   4      : 50  
 3rd Qu.:-1.7744   41     : 18   5      : 39  
 Max.   :-0.9684   28     : 15   6      : 33  
                   (Other):379   (Other):157  
\end{Soutput}
\end{Schunk}

The printing consists on four sections :

\begin{itemize}
\item \texttt{indexes} indicates the names of the index variables,
\item \texttt{panel dimensions} gives information about the dimension
  of the panel,
\item \texttt{Time/individual variation} indicates whether some
  variables have only individual or time variation,
\item \texttt{Descriptive statistics} gives descriptive statistics
  about the variables.
\end{itemize}

\subsection{Reading the data from a text file}

\texttt{pread.table} reads panel data from a text file, with the
following syntax : 

\begin{verbatim}
pread.table("c:/mes documents/essai/mydata.txt",
         "firm","year","dataname",header=T,sep=";",dec=",")
\end{verbatim}

The arguments of  \texttt{pread.table} are :

\begin{itemize}
\item the text file,
\item \texttt{id} : the individual index,
\item \texttt{time} : the time index,
\item \texttt{name} : the name under which the  \texttt{pdata.frame}
  will be stored  (if \texttt{NULL}, the name of the \texttt{pdata.frame}
  is the name of the file without the path and the extension),
\item further arguments that will be passed to  \texttt{read.table}.
\end{itemize}




\section{Model estimation}

A panel model is estimated with the \texttt{plm} function.

\subsection{Basic use of plm}

There are two ways to use \texttt{plm} : the first one is to estimate
a list of models (the default behavior), the second to estimate just one model.
In the first case, the estimated models are :

\begin{itemize}
\item the fixed effects model (\texttt{within}),
\item the pooling model (\texttt{pooling}),
\item the between model (\texttt{between}),
\item the error components model (\texttt{random}).
\end{itemize}

The basic use of \texttt{plm} is to indicate the model formula and the \texttt{pdata.frame}
\footnote{The following example is from \textsc{Baltagi} (2001), pp. 25--28.} :

\begin{Schunk}
\begin{Sinput}
> zz <- plm(log(gsp) ~ log(pcap) + log(pc) + log(emp) + unemp, 
+     data = pprod)
\end{Sinput}
\end{Schunk}

The result of the estimation is stored in a \texttt{plms} object which
is a list of 4 estimated models, each of them being objects of class \texttt{plm}.
Each individual model can be easily extracted :

\begin{Schunk}
\begin{Sinput}
> zzwith <- zz$within
\end{Sinput}
\end{Schunk}

A particular model to be estimated may also be indicated by filling
the \texttt{model} argument of \texttt{plm}.

\begin{Schunk}
\begin{Sinput}
> zzra <- plm(log(gsp) ~ log(pcap) + log(pc) + log(emp) + unemp, 
+     data = pprod, model = "random")
\end{Sinput}
\end{Schunk}

Objects of class \texttt{plm} and \texttt{plms} have a \texttt{print}
method. 

\begin{Schunk}
\begin{Sinput}
> print(zzra)
\end{Sinput}
\begin{Soutput}
Model Formula: log(gsp) ~ log(pcap) + log(pc) + log(emp) + unemp

Coefficients:
(intercept)   log(pcap)     log(pc)    log(emp)       unemp 
  2.1354110   0.0044386   0.3105484   0.7296705  -0.0061725 
\end{Soutput}
\end{Schunk}

There is also a \texttt{summary} method :

\begin{itemize}
\item for  \texttt{plms} objects, coefficients and standard errors
  of the fixed effects and the error components models are printed,
\item for  \texttt{plm} object, the table of coefficients and some
  statistics are printed.
\end{itemize}


\begin{Schunk}
\begin{Sinput}
> summary(zz)
\end{Sinput}
\begin{Soutput}
______________________________________________________________________ 
_________________________ Model Description __________________________
Oneway (individual) effect

Model Formula        : log(gsp) ~ log(pcap) + log(pc) + log(emp) + 
                           unemp
______________________________________________________________________ 
__________________________ Panel Dimensions __________________________
Balanced Panel
Number of Individuals        :  48
Number of Time Obserbations  :  17
Total Number of Observations :  816
______________________________________________________________________ 
____________________________ Coefficients ____________________________
                 within         wse      random    rse
(intercept)           .           .  2.13541100 0.1335
log(pcap)   -0.02614965  0.02900158  0.00443859 0.0234
log(pc)      0.29200693  0.02511967  0.31054843 0.0198
log(emp)     0.76815947  0.03009174  0.72967053 0.0249
unemp       -0.00529774  0.00098873 -0.00617247 0.0009
______________________________________________________________________ 
_______________________________ Tests ________________________________
Hausman Test                   : chi2(4) = 9.525416 (p.value=0.04922762)
F Test                         : F(47,764) = 75.8204 (p.value=0)
Lagrange Multiplier Test       : chi2(1) = 4134.961 (p.value=0)
______________________________________________________________________ 
\end{Soutput}
\begin{Sinput}
> summary(zzra)
\end{Sinput}
\begin{Soutput}
______________________________________________________________________ 
_________________________ Model Description __________________________
Oneway (individual) effect
Random Effect Model (Swamy-Arora's transformation)
Model Formula             : log(gsp) ~ log(pcap) + log(pc) + 
                                log(emp) + unemp
______________________________________________________________________ 
__________________________ Panel Dimensions __________________________
Balanced Panel
Number of Individuals        :  48
Number of Time Obserbations  :  17
Total Number of Observations :  816
______________________________________________________________________ 
______________________________ Effects _______________________________
                    var   std.dev  share
idiosyncratic 0.0014544 0.0381371 0.1754
individual    0.0068377 0.0826905 0.8246
theta   :  0.88884  
______________________________________________________________________ 
_____________________________ Residuals ______________________________
     Min.   1st Qu.    Median      Mean   3rd Qu.      Max. 
-1.07e-01 -2.46e-02 -2.37e-03 -9.93e-19  2.17e-02  2.00e-01 
______________________________________________________________________ 
____________________________ Coefficients ____________________________
               Estimate  Std. Error z-value  Pr(>|z|)    
(intercept)  2.13541100  0.13346149 16.0002 < 2.2e-16 ***
log(pcap)    0.00443859  0.02341732  0.1895    0.8497    
log(pc)      0.31054843  0.01980475 15.6805 < 2.2e-16 ***
log(emp)     0.72967053  0.02492022 29.2803 < 2.2e-16 ***
unemp       -0.00617247  0.00090728 -6.8033 1.023e-11 ***
---
Signif. codes:  0 ‘***’ 0.001 ‘**’ 0.01 ‘*’ 0.05 ‘.’ 0.1 ‘ ’ 1 
______________________________________________________________________ 
_________________________ Overall Statistics _________________________
Total Sum of Squares       : 29.209
Sum of Squares Residuals   : 1.1879
Rsq                        : 0.95933
F                          : 4782.77
P(F>0)                     : 8.76231e-08
______________________________________________________________________ 
\end{Soutput}
\end{Schunk}

For a \texttt{random} model, the \texttt{summary} method gives
information about the variance of the components of the errors.

\texttt{plm}'s can be updated using the \texttt{update} method :

\begin{Schunk}
\begin{Sinput}
> zzwithmod <- update(zzwith, . ~ . - unemp - log(emp) + emp)
> zzmod <- update(zz, . ~ . - unemp - log(emp) + emp)
> summary(zzwithmod)
\end{Sinput}
\begin{Soutput}
______________________________________________________________________ 
_________________________ Model Description __________________________
Oneway (individual) effect

Model Formula        : log(gsp) ~ log(pcap) + log(pc) + emp
______________________________________________________________________ 
__________________________ Panel Dimensions __________________________
Balanced Panel
Number of Individuals        :  48
Number of Time Obserbations  :  17
Total Number of Observations :  816
______________________________________________________________________ 
____________________________ Coefficients ____________________________
                within        wse     random       rse
(intercept)          .          . 7.1982e-01    0.1846
log(pcap)   1.7888e-01 4.0690e-02 3.4357e-01    0.0322
log(pc)     6.9975e-01 2.9154e-02 6.0369e-01    0.0256
emp         3.7909e-05 8.7824e-06 5.0924e-05 8.218e-06
______________________________________________________________________ 
_______________________________ Tests ________________________________
Hausman Test                   : chi2(3) = 80.35868 (p.value=0)
F Test                         : F(47,765) = 101.9109 (p.value=0)
Lagrange Multiplier Test       : chi2(1) = 4355.292 (p.value=0)
______________________________________________________________________ 
\end{Soutput}
\end{Schunk}

Fixed effects may be extracted easily from a \texttt{plms} or a
\texttt{plm} object using  \texttt{FE} :

\begin{Schunk}
\begin{Sinput}
> FE(zzmod)
\end{Sinput}
\begin{Soutput}
       ALABAMA        ARIZONA       ARKANSAS     CALIFORNIA       COLORADO 
     1.1717531      1.3062389      1.1877004      1.6191982      1.4582149 
   CONNECTICUT       DELAWARE        FLORIDA        GEORGIA          IDAHO 
     1.7060341      1.2035746      1.5564969      1.4460171      1.1002049 
      ILLINOIS        INDIANA           IOWA         KANSAS       KENTUCKY 
     1.5496106      1.3451714      1.2323038      1.1735476      1.3492604 
     LOUISIANA          MAINE       MARYLAND  MASSACHUSETTS       MICHIGAN 
     1.1652834      1.2659480      1.6011871      1.7384231      1.5290312 
     MINNESOTA    MISSISSIPPI       MISSOURI        MONTANA       NEBRASKA 
     1.3654287      1.1545345      1.4809262      0.7960951      1.0905033 
        NEVADA  NEW_HAMPSHIRE     NEW_JERSEY     NEW_MEXICO       NEW_YORK 
     1.0627992      1.4138235      1.7420589      1.0925399      1.6694387 
NORTH_CAROLINA   NORTH_DAKOTA           OHIO       OKLAHOMA         OREGON 
     1.5048751      0.7663694      1.4985974      1.2784660      1.3345094 
  PENNSYLVANIA   RHODE_ISLAND SOUTH_CAROLINA   SOUTH_DAKOTA       TENNESSE 
     1.4972243      1.5948140      1.2344011      0.8705826      1.3123010 
         TEXAS           UTAH        VERMONT       VIRGINIA     WASHINGTON 
     1.3230328      1.2464927      1.1804339      1.6175357      1.3492922 
 WEST_VIRGINIA      WISCONSIN        WYOMING 
     1.0129871      1.4860561      0.7842841 
\end{Soutput}
\end{Schunk}

\subsection{Options for the random effect model}

The random effect model is obtained as a linear estimation on
quasi--differentiated  data. The parameter of this transformation is
obtained using preliminary estimations. Four estimators of this
parameter are available, depending on the value of the argument \texttt{theta.method}  :

\begin{itemize}
\item \texttt{swar} : from \textsc{Swamy} and \textsc{Arora}
  (1972), the default value,
\item \texttt{walhus} : from \textsc{Wallace} and \textsc{Hussain} (1969),
\item \texttt{amemiya} : from \textsc{Amemiyia} (1971),
\item \texttt{nerlove} : from \textsc{Nerlove} (1971).
\end{itemize}

For exemple, to use the \texttt{amemiya} estimator :

\begin{Schunk}
\begin{Sinput}
> zzra <- plm(log(gsp) ~ log(pcap) + log(pc) + log(emp) + unemp, 
+     data = pprod, model = "random", theta.method = "amemiya")
\end{Sinput}
\end{Schunk}


\subsection{Choosing  the effects}

The default behavior of \texttt{plm} is to introduce individual
effects. Using the \texttt{effect} argument, one may also introduce :

\begin{itemize}
\item time effects (\texttt{effect="time"}),
\item individual and time effects (\texttt{effect="twoways"}).
\end{itemize}

For example, to estimate a two--ways effect model for the
\texttt{Grunfeld} data :

\begin{Schunk}
\begin{Sinput}
> data(Grunfeld)
> pdata.frame(Grunfeld, "firm", "year")
> z <- plm(inv ~ value + capital, data = Grunfeld, effect = "twoways", 
+     theta.method = "amemiya")
> summary(z$random)
\end{Sinput}
\begin{Soutput}
______________________________________________________________________ 
_________________________ Model Description __________________________
Twoways effects
Random Effect Model (Swamy-Arora's transformation)
Model Formula             : inv ~ value + capital
______________________________________________________________________ 
__________________________ Panel Dimensions __________________________
Balanced Panel
Number of Individuals        :  10
Number of Time Obserbations  :  20
Total Number of Observations :  200
______________________________________________________________________ 
______________________________ Effects _______________________________
                   var  std.dev  share
idiosyncratic 2675.426   51.725 0.2738
individual    7095.252   84.233 0.7262
time             0.000    0.000 0.0000
theta  : 0.86397 (id) 0 (time) 0 (total)
______________________________________________________________________ 
_____________________________ Residuals ______________________________
     Min.   1st Qu.    Median      Mean   3rd Qu.      Max. 
-1.77e+02 -1.98e+01  4.60e+00  8.77e-16  1.95e+01  2.53e+02 
______________________________________________________________________ 
____________________________ Coefficients ____________________________
              Estimate Std. Error z-value Pr(>|z|)    
(intercept) -57.865377  29.393359 -1.9687  0.04899 *  
value         0.109790   0.010528 10.4285  < 2e-16 ***
capital       0.308190   0.017171 17.9483  < 2e-16 ***
---
Signif. codes:  0 ‘***’ 0.001 ‘**’ 0.01 ‘*’ 0.05 ‘.’ 0.1 ‘ ’ 1 
______________________________________________________________________ 
_________________________ Overall Statistics _________________________
Total Sum of Squares       : 2376000
Sum of Squares Residuals   : 547910
Rsq                        : 0.7694
F                          : 328.647
P(F>0)                     : 0.0030381
______________________________________________________________________ 
\end{Soutput}
\end{Schunk}

In the ``effects'' section of the result is printed now the variance
of the three elements of the error term and the three parameters used
in the transformation. 

The two--ways effect model is for the moment only available for
balanced panels.


\subsection{Hausman--Taylor's model}

\textsc{Hausman}--\textsc{Taylor}'s model may be estimated with \texttt{plm}
by equating the \texttt{model} argument to  \texttt{"ht"} and
filling the second argument \texttt{instruments} with a formula
indicating the variables used as instruments.


\begin{Schunk}
\begin{Sinput}
> data(Wages)
> pdata.frame(Wages, 595)
> form = lwage ~ wks + south + smsa + married + exp + I(exp^2) + 
+     bluecol + ind + union + sex + black + ed
> ht = plm(form, ~sex + black + bluecol + south + smsa + ind, data = Wages, 
+     model = "ht")
> summary(ht)
\end{Sinput}
\begin{Soutput}
______________________________________________________________________ 
_________________________ Model Description __________________________
Oneway (individual) effect
Hausman-Taylor Model
Model Formula             : lwage ~ wks + south + smsa + married + 
                                exp + I(exp^2) + bluecol + ind + 
                                union + sex + black + ed
Instrumental Variables    : ~sex + black + bluecol + south + 
                                smsa + ind
Time--Varying Variables    
    exogenous variables   :  bluecolyes,southyes,smsayes,ind 
    endogenous variables  :  wks,marriedyes,exp,I(exp^2),unionyes 
Time--Invariant Variables  
    exogenous variables   :  sexmale,blackyes 
    endogenous variables  :  ed 
______________________________________________________________________ 
__________________________ Panel Dimensions __________________________
Balanced Panel
Number of Individuals        :  595
Number of Time Obserbations  :  7
Total Number of Observations :  4165
______________________________________________________________________ 
______________________________ Effects _______________________________
                   var  std.dev  share
idiosyncratic 0.023044 0.151803 0.0253
individual    0.886993 0.941803 0.9747
theta   :  0.93919  
______________________________________________________________________ 
_____________________________ Residuals ______________________________
     Min.   1st Qu.    Median      Mean   3rd Qu.      Max. 
-1.92e+00 -7.07e-02  6.57e-03 -2.46e-17  7.97e-02  2.03e+00 
______________________________________________________________________ 
____________________________ Coefficients ____________________________
               Estimate  Std. Error z-value  Pr(>|z|)    
(intercept)  2.7818e+00  3.0768e-01  9.0411 < 2.2e-16 ***
wks          8.3740e-04  5.9981e-04  1.3961   0.16268    
southyes     7.4398e-03  3.1959e-02  0.2328   0.81592    
smsayes     -4.1833e-02  1.8960e-02 -2.2064   0.02736 *  
marriedyes  -2.9851e-02  1.8982e-02 -1.5726   0.11582    
exp          1.1313e-01  2.4713e-03 45.7795 < 2.2e-16 ***
I(exp^2)    -4.1886e-04  5.4605e-05 -7.6709 1.710e-14 ***
bluecolyes  -2.0705e-02  1.3783e-02 -1.5022   0.13304    
ind          1.3604e-02  1.5239e-02  0.8927   0.37202    
unionyes     3.2771e-02  1.4910e-02  2.1979   0.02796 *  
sexmale      1.3092e-01  1.2667e-01  1.0335   0.30135    
blackyes    -2.8575e-01  1.5572e-01 -1.8350   0.06651 .  
ed           1.3794e-01  2.1251e-02  6.4912 8.518e-11 ***
---
Signif. codes:  0 ‘***’ 0.001 ‘**’ 0.01 ‘*’ 0.05 ‘.’ 0.1 ‘ ’ 1 
______________________________________________________________________ 
_________________________ Overall Statistics _________________________
Total Sum of Squares       : 243.04
Sum of Squares Residuals   : 95.947
Rsq                        : 0.60522
F                          : 530.318
P(F>0)                     : 2.88658e-15
______________________________________________________________________ 
\end{Soutput}
\end{Schunk}

\subsection{Instrumental variables estimation}

One or all of the models may be estimated using instrumental variables
by indicating the list of the instrumental variables. This can be done
using one of the two following techniques :

\begin{itemize}
\item specifying the total list of instruments  (using the
  \texttt{instruments} argument of \texttt{plm}),
\item specifying, on the one hand the external instruments in the argument
  \texttt{instrument} and on  the other hand the variables of the
  model that are assumed to be endogenous in the argument \texttt{endog}.
\end{itemize}

The instrumental variables estimator used may be indicated with the
\texttt{inst.method} argument :
\begin{itemize}
\item \texttt{bvk}, from  \textsc{Balestra} et
  \textsc{Varadharajan--Krishnakumar} (1987), the default value,
\item \texttt{baltagi}, from \textsc{Baltagi} (1981).
\end{itemize}

We illustrate instrumental variables estimation with the
\texttt{Crime} data\footnote{See
  \textsc{Baltagi} (2001), pp.119--120.}. 
The same estimation is done using the first syntax  (\texttt{cr1}) and
the second (\texttt{cr2}). The  \texttt{prbarr} and \texttt{polpc}
variables are
assumed to be endogenous and there are two external instruments \texttt{taxpc} and \texttt{mix} :

\begin{Schunk}
\begin{Sinput}
> data(Crime)
> pdata.frame(Crime, "county", "year")
> form = log(crmrte) ~ log(prbarr) + log(polpc) + log(prbconv) + 
+     log(prbpris) + log(avgsen) + log(density) + log(wcon) + log(wtuc) + 
+     log(wtrd) + log(wfir) + log(wser) + log(wmfg) + log(wfed) + 
+     log(wsta) + log(wloc) + log(pctymle) + log(pctmin) + region + 
+     smsa + year
> inst = ~log(prbconv) + log(prbpris) + log(avgsen) + log(density) + 
+     log(wcon) + log(wtuc) + log(wtrd) + log(wfir) + log(wser) + 
+     log(wmfg) + log(wfed) + log(wsta) + log(wloc) + log(pctymle) + 
+     log(pctmin) + region + smsa + log(taxpc) + log(mix) + year
> inst2 = ~log(taxpc) + log(mix)
> endog = ~log(prbarr) + log(polpc)
> cr = plm(form, data = Crime)
> cr1 = plm(form, inst, data = Crime)
> cr2 = plm(form, inst2, endog, data = Crime)
> summary(cr2$random)
\end{Sinput}
\begin{Soutput}
______________________________________________________________________ 
_________________________ Model Description __________________________
Oneway (individual) effect
Random Effect Model (Swamy-Arora's transformation)
Instrumental variable estimation (Balestra-Varadharajan-Krishnakumar's transformation)
Model Formula             : log(crmrte) ~ log(prbarr) + log(polpc) + 
                                log(prbconv) + log(prbpris) + 
                                log(avgsen) + log(density) + 
                                log(wcon) + log(wtuc) + log(wtrd) + 
                                log(wfir) + log(wser) + log(wmfg) + 
                                log(wfed) + log(wsta) + log(wloc) + 
                                log(pctymle) + log(pctmin) + 
                                region + smsa + year
Endogenous Variables    : ~log(prbarr) + log(polpc)
Instrumental Variables    : ~log(taxpc) + log(mix)
______________________________________________________________________ 
__________________________ Panel Dimensions __________________________
Balanced Panel
Number of Individuals        :  90
Number of Time Obserbations  :  7
Total Number of Observations :  630
______________________________________________________________________ 
______________________________ Effects _______________________________
                   var  std.dev share
idiosyncratic 0.022269 0.149228 0.326
individual    0.046036 0.214561 0.674
theta   :  0.74576  
______________________________________________________________________ 
_____________________________ Residuals ______________________________
     Min.   1st Qu.    Median      Mean   3rd Qu.      Max. 
-5.02e+00 -4.76e-01  2.73e-02  7.11e-16  5.26e-01  3.19e+00 
______________________________________________________________________ 
____________________________ Coefficients ____________________________
                Estimate Std. Error z-value  Pr(>|z|)    
(intercept)   -0.4538241  1.7029840 -0.2665  0.789864    
log(prbarr)   -0.4141200  0.2210540 -1.8734  0.061015 .  
log(polpc)     0.5049285  0.2277811  2.2167  0.026642 *  
log(prbconv)  -0.3432383  0.1324679 -2.5911  0.009567 ** 
log(prbpris)  -0.1900437  0.0733420 -2.5912  0.009564 ** 
log(avgsen)   -0.0064374  0.0289406 -0.2224  0.823977    
log(density)   0.4343519  0.0711528  6.1045 1.031e-09 ***
log(wcon)     -0.0042963  0.0414225 -0.1037  0.917392    
log(wtuc)      0.0444572  0.0215449  2.0635  0.039068 *  
log(wtrd)     -0.0085626  0.0419822 -0.2040  0.838387    
log(wfir)     -0.0040302  0.0294565 -0.1368  0.891175    
log(wser)      0.0105604  0.0215822  0.4893  0.624620    
log(wmfg)     -0.2017917  0.0839423 -2.4039  0.016220 *  
log(wfed)     -0.2134634  0.2151074 -0.9924  0.321023    
log(wsta)     -0.0601083  0.1203146 -0.4996  0.617362    
log(wloc)      0.1835137  0.1396721  1.3139  0.188884    
log(pctymle)  -0.1458448  0.2268137 -0.6430  0.520214    
log(pctmin)    0.1948760  0.0459409  4.2419 2.217e-05 ***
regionwest    -0.2281780  0.1010317 -2.2585  0.023916 *  
regioncentral -0.1987675  0.0607510 -3.2718  0.001068 ** 
smsayes       -0.2595423  0.1499780 -1.7305  0.083535 .  
year82         0.0132140  0.0299923  0.4406  0.659518    
year83        -0.0847676  0.0320008 -2.6489  0.008075 ** 
year84        -0.1062004  0.0387893 -2.7379  0.006184 ** 
year85        -0.0977398  0.0511685 -1.9102  0.056113 .  
year86        -0.0719390  0.0605821 -1.1875  0.235045    
year87        -0.0396520  0.0758537 -0.5227  0.601153    
---
Signif. codes:  0 ‘***’ 0.001 ‘**’ 0.01 ‘*’ 0.05 ‘.’ 0.1 ‘ ’ 1 
______________________________________________________________________ 
_________________________ Overall Statistics _________________________
Total Sum of Squares       : 1354.7
Sum of Squares Residuals   : 557.64
Rsq                        : 0.58836
F                          : 33.1494
P(F>0)                     : 7.77156e-16
______________________________________________________________________ 
\end{Soutput}
\end{Schunk}


\subsection{Variable coefficients model}

If there is enough time observations for each individual, the model
may be estimate for each individual. The \texttt{nopool} function
provide this kind of estimation. It can be done using :

\begin{itemize}
\item directly the \texttt{nopool} function,
\item \texttt{plm} with the argument \texttt{np} fixed to \texttt{TRUE}.
\end{itemize}

With the \texttt{Grunfeld} data, we get :

\begin{Schunk}
\begin{Sinput}
> znp <- nopool(inv ~ value + capital, data = Grunfeld)
\end{Sinput}
\end{Schunk}

or 

\begin{Schunk}
\begin{Sinput}
> z <- plm(inv ~ value + capital, data = Grunfeld, np = TRUE)
> znp <- z$nopool
> print(znp)
\end{Sinput}
\begin{Soutput}
    (intercept)       value     capital
1  -149.7824533 0.119280833 0.371444807
2   -49.1983219 0.174856015 0.389641889
3    -9.9563065 0.026551189 0.151693870
4    -6.1899605 0.077947821 0.315718185
5    22.7071160 0.162377704 0.003101737
6    -8.6855434 0.131454842 0.085374274
7    -4.4995344 0.087527198 0.123781407
8    -0.5093902 0.052894126 0.092406492
9    -7.7228371 0.075387943 0.082103558
10    0.1615186 0.004573432 0.437369190
\end{Soutput}
\begin{Sinput}
> summary(znp)
\end{Sinput}
\begin{Soutput}
  (intercept)           value             capital        
 Min.   :-149.782   Min.   :0.004573   Min.   :0.003102  
 1st Qu.:  -9.639   1st Qu.:0.058518   1st Qu.:0.087132  
 Median :  -6.956   Median :0.082738   Median :0.137738  
 Mean   : -21.368   Mean   :0.091285   Mean   :0.205264  
 3rd Qu.:  -1.507   3rd Qu.:0.128411   3rd Qu.:0.357513  
 Max.   :  22.707   Max.   :0.174856   Max.   :0.437369  
\end{Soutput}
\end{Schunk}


The result is an object of class  \texttt{nopool}. The \texttt{print}
method presents the coefficients estimated for each individual. The
\texttt{summary} method gives descriptive statistics for these coefficients.

\subsection{Unbalanced panel}

\texttt{plm} offers limited support for unbalanced panels. The
following example is based on the \texttt{Hedonic} data\footnote{See \textsc{Baltagi}
  (2001), p. 174.}:

\begin{Schunk}
\begin{Sinput}
> form = mv ~ crim + zn + indus + chas + nox + rm + age + dis + 
+     rad + tax + ptratio + blacks + lstat
> ba = plm(form, data = Hedonic)
> summary(ba$random)
\end{Sinput}
\begin{Soutput}
______________________________________________________________________ 
_________________________ Model Description __________________________
Oneway (individual) effect
Random Effect Model (Swamy-Arora's transformation)
Model Formula             : mv ~ crim + zn + indus + chas + nox + 
                                rm + age + dis + rad + tax + 
                                ptratio + blacks + lstat
______________________________________________________________________ 
__________________________ Panel Dimensions __________________________
Unbalanced Panel
Number of Individuals        :  92
Number of Time Obserbations  :  from 1 to 30
Total Number of Observations :  506
______________________________________________________________________ 
______________________________ Effects _______________________________
                   var  std.dev share
idiosyncratic 0.016965 0.130249 0.502
individual    0.016832 0.129738 0.498
theta  : 
   Min. 1st Qu.  Median    Mean 3rd Qu.    Max. 
 0.2915  0.5904  0.6655  0.6499  0.7447  0.8197 
______________________________________________________________________ 
_____________________________ Residuals ______________________________
     Min.   1st Qu.    Median      Mean   3rd Qu.      Max. 
-0.641000 -0.066100 -0.000519 -0.001990  0.069800  0.527000 
______________________________________________________________________ 
____________________________ Coefficients ____________________________
               Estimate  Std. Error  z-value  Pr(>|z|)    
(intercept)  9.6778e+00  2.0714e-01  46.7207 < 2.2e-16 ***
crim        -7.2338e-03  1.0346e-03  -6.9921 2.707e-12 ***
zn           3.9575e-05  6.8778e-04   0.0575 0.9541153    
indus        2.0794e-03  4.3403e-03   0.4791 0.6318706    
chasyes     -1.0591e-02  2.8960e-02  -0.3657 0.7145720    
nox         -5.8630e-03  1.2455e-03  -4.7074 2.509e-06 ***
rm           9.1773e-03  1.1792e-03   7.7828 7.105e-15 ***
age         -9.2715e-04  4.6468e-04  -1.9952 0.0460159 *  
dis         -1.3288e-01  4.5683e-02  -2.9088 0.0036279 ** 
rad          9.6863e-02  2.8350e-02   3.4168 0.0006337 ***
tax         -3.7472e-04  1.8902e-04  -1.9824 0.0474298 *  
ptratio     -2.9723e-02  9.7538e-03  -3.0473 0.0023089 ** 
blacks       5.7506e-01  1.0103e-01   5.6920 1.256e-08 ***
lstat       -2.8514e-01  2.3855e-02 -11.9533 < 2.2e-16 ***
---
Signif. codes:  0 ‘***’ 0.001 ‘**’ 0.01 ‘*’ 0.05 ‘.’ 0.1 ‘ ’ 1 
______________________________________________________________________ 
_________________________ Overall Statistics _________________________
Total Sum of Squares       : 893.08
Sum of Squares Residuals   : 8.6843
Rsq                        : 0.99028
F                          : 3854.18
P(F>0)                     : 0
______________________________________________________________________ 
\end{Soutput}
\end{Schunk}
\section{Tests}


\subsection{Tests of poolability}

\texttt{pooltest} tests the hypothesis that the same coefficients
apply to each individual. It is a standard F test, based on the
comparison of a model obtained for the full sample and a model based
on the estimation of an equation for each individual. The main
argument of \texttt{pooltest} is a \texttt{plm} object. If the model
has been estimated with the argument  \texttt{np=F}, one has to
indicate a second argument of class  \texttt{nopool}.
A third argument  \texttt{effect} should be fixed to \texttt{FALSE} if
the intercepts are assumed to be identical (the default value) or \texttt{TRUE} if
not\footnote{The following examples are from 
  \textsc{Baltagi} (2001), pp. 57--58.}.

\begin{Schunk}
\begin{Sinput}
> form = inv ~ value + capital
> znp = nopool(form, data = Grunfeld)
> zplm = plm(form, data = Grunfeld)
> pooltest(zplm, znp)
\end{Sinput}
\begin{Soutput}
	F statistic

data:  zplm 
F = 27.7486, df1 = 27, df2 = 170, p-value < 2.2e-16
\end{Soutput}
\begin{Sinput}
> pooltest(zplm, znp, effect = T)
\end{Sinput}
\begin{Soutput}
	F statistic

data:  zplm 
F = 5.7805, df1 = 18, df2 = 170, p-value = 1.219e-10
\end{Soutput}
\begin{Sinput}
> z = plm(form, data = Grunfeld, effect = "time", np = TRUE)
> pooltest(z, effect = F)
\end{Sinput}
\begin{Soutput}
	F statistic

data:  z 
F = 1.1204, df1 = 57, df2 = 140, p-value = 0.2928
\end{Soutput}
\end{Schunk}


\subsection{Tests for individual and time effects}



\subsubsection{Lagrange multiplier tests}

\texttt{plmtest} implements tests of individual or/and time effects  based on the results
of the pooling model. It's main argument is a
\texttt{plm} object (the result of a pooling model) or a
\texttt{plms} object.

Two additional arguments can be added to indicate the kind of test to
be computed. The argument \texttt{type} is whether :

\begin{itemize}
\item \texttt{bp} : \textsc{Breusch--Pagan} (1980), the default value,
\item \texttt{honda} : \textsc{Honda} (1985),
\item \texttt{kw} : \textsc{King} and \textsc{Wu} (1997).
\end{itemize}

The effects tested are indicated with the  \texttt{effect} argument :

\begin{itemize}
\item \texttt{individual} for individual effects  (the default value),
\item \texttt{time} for time effects,
\item \texttt{twoways} for individuals and time effects.
\end{itemize}

Some examples of the use of \texttt{plmtest} are shown below\footnote{See \textsc{Baltagi} (2001), p. 65.}:

\begin{Schunk}
\begin{Sinput}
> library(Ecdat)
> g <- plm(inv ~ value + capital, data = Grunfeld)
> plmtest(g)
\end{Sinput}
\begin{Soutput}
	Lagrange Multiplier Test - individual effects (Breush-Pagan)

data:  g 
chi2 = 798.1615, df = 1, p-value < 2.2e-16
\end{Soutput}
\begin{Sinput}
> plmtest(g, effect = "time")
\end{Sinput}
\begin{Soutput}
	Lagrange Multiplier Test - time effects (Breush-Pagan)

data:  g 
chi2 = 6.4539, df = 1, p-value = 0.01107
\end{Soutput}
\begin{Sinput}
> plmtest(g, type = "honda")
\end{Sinput}
\begin{Soutput}
	Lagrange Multiplier Test - individual effects (Honda)

data:  g 
normal = 28.2518, p-value < 2.2e-16
\end{Soutput}
\begin{Sinput}
> plmtest(g, type = "ghm", effect = "twoways")
\end{Sinput}
\begin{Soutput}
	Lagrange Multiplier Test - two-ways effects (Gourierroux, Holly and
	Monfort)

data:  g 
chi2 = 798.1615, df = 2, p-value < 2.2e-16
\end{Soutput}
\begin{Sinput}
> plmtest(g, type = "kw", effect = "twoways")
\end{Sinput}
\begin{Soutput}
	Lagrange Multiplier Test - two-ways effects (King and Wu)

data:  g 
normal = 21.8322, df = 2, p-value < 2.2e-16
\end{Soutput}
\end{Schunk}

\subsubsection{F tests}

\texttt{pFtest} computes F tests of effects based on the comparison of
the  \texttt{within} and the \texttt{pooling} models. Its arguments
are whether a \texttt{plms} object or two \texttt{plm} objects
(the results of a  \texttt{pooling} and a \texttt{within} model).
Some examples of the use of \texttt{pFtest} are shown below\footnote{Voir \textsc{Baltagi} (2001),
  p. 65.}:

\begin{Schunk}
\begin{Sinput}
> library(Ecdat)
> gi <- plm(inv ~ value + capital, data = Grunfeld)
> gt <- plm(inv ~ value + capital, data = Grunfeld, effect = "time")
> gd <- plm(inv ~ value + capital, data = Grunfeld, effect = "twoways")
> pFtest(gi)
\end{Sinput}
\begin{Soutput}
	F test for effects

data:  gi 
F = 49.1766, df1 = 9, df2 = 188, p-value < 2.2e-16
\end{Soutput}
\begin{Sinput}
> pFtest(gi$within, gi$pooling)
\end{Sinput}
\begin{Soutput}
	F test for effects

data:  gi$within and gi$pooling 
F = 49.1766, df1 = 9, df2 = 188, p-value < 2.2e-16
\end{Soutput}
\begin{Sinput}
> pFtest(gt)
\end{Sinput}
\begin{Soutput}
	F test for effects

data:  gt 
F = 0.5229, df1 = 9, df2 = 188, p-value = 0.8569
\end{Soutput}
\begin{Sinput}
> pFtest(gd)
\end{Sinput}
\begin{Soutput}
	F test for effects

data:  gd 
F = 17.4031, df1 = 28, df2 = 169, p-value < 2.2e-16
\end{Soutput}
\end{Schunk}



\subsection{Hausman's test}

\texttt{phtest} computes the \textsc{Hausman}'s test which is based on
the  comparison of two models. It's main argument may be :

\begin{itemize}
\item a \texttt{plms} object. In this case, the two models used in the
  test are the \texttt{within} and the \texttt{random} models (the
  most usual case with panel data),
\item two \texttt{plm} objects.
\end{itemize}


Some examples of the use of \texttt{phtest} are shown below
\footnote{See \textsc{Baltagi} (2001), p. 71.}:


\begin{Schunk}
\begin{Sinput}
> g <- plm(inv ~ value + capital, data = Grunfeld)
> phtest(g)
\end{Sinput}
\begin{Soutput}
	Hausman Test

data:  g 
chi2 = 2.3304, df = 2, p-value = 0.3119
\end{Soutput}
\begin{Sinput}
> phtest(g$between, g$random)
\end{Sinput}
\begin{Soutput}
	Hausman Test

data:  g$between and g$random 
chi2 = 2.1314, df = 3, p-value = 0.5456
\end{Soutput}
\end{Schunk}


\section{Bibiographie}

\setlength{\parindent}{0em}
\setlength{\parskip}{0.4cm}

  Amemiyia, T. (1971), The estimation of the variances in a
  variance--components model, \emph{International Economic Review}, 12,
  pp.1--13.

  Balestra, P. and J. Varadharajan--Krishnakumar (1987), Full
  information estimations of a system of simultaneous equations with
  error components structure, \emph{Econometric Theory}, 3, pp.223--246.
  
  Baltagi, B.H. (1981), Simultaneous equations with error components,
  \emph{Journal of econometrics}, 17, pp.21--49.
  
  Baltagi, B.H. (2001) \emph{Econometric Analysis of Panel Data}. John
  Wiley and sons. ltd.

  Breusch, T.S. and A.R. Pagan (1980), The Lagrange multiplier test and
  its applications to model specification in econometrics, \emph{Review
    of Economic Studies}, 47, pp.239--253.

  Gourieroux, C., A. Holly and A. Monfort (1982), Likelihood ratio test,
  Wald test, and Kuhn--Tucker test in linear models with inequality
  constraints on the regression parameters, \emph{Econometrica}, 50,
  pp.63--80.

  Hausman, G. (1978), Specification tests in econometrics,
  \emph{Econometrica}, 46, pp.1251--1271.

  Hausman, J.A. and W.E. Taylor (1981), Panel data and unobservable
  individual effects, \emph{Econometrica}, 49, pp.1377--1398.
  
  Honda, Y. (1985), Testing the error components model with non--normal
  disturbances, \emph{Review of Economic Studies}, 52, pp.681--690.

  King, M.L. and P.X. Wu (1997), Locally optimal one--sided tests for
  multiparameter hypotheses, \emph{Econometric Reviews}, 33,
  pp.523--529.

  Nerlove, M. (1971), Further evidence on the estimation of dynamic
  economic relations from a time--series of cross--sections,
  \emph{Econometrica}, 39, pp.359--382.

  Swamy, P.A.V.B. and S.S. Arora (1972), The exact finite sample
  properties of the estimators of coefficients in the error components
  regression models, \emph{Econometrica}, 40, pp.261--275.

  Wallace, T.D. and A. Hussain (1969), The use of error components
  models in combining cross section with time series data,
  \emph{Econometrica}, 37(1), pp.55--72.

\end{document}


