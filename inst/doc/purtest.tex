\documentclass{article}

\begin{document}


\section{Preliminary results}

We consider the following model :

$$
y_{it} = \delta y_{it-1} + \sum_{L=1}^{p_i} \theta_i \Delta
y_{it-L}+\alpha_{mi} d_{mt}+\epsilon_{it}
$$

The unit root hypothesis is $\rho = 1$. The model can be rewriten in
difference :

$$
\Delta y_{it} = \rho y_{it-1} + \sum_{L=1}^{p_i} \theta_i \Delta
y_{it-L}+\alpha_{mi} d_{mt}+\epsilon_{it}
$$

So that the unit-root hypothesis is now $\rho = 0$.

Some of the unit-root tests for panel data are based on preliminary
results obtained by runing the above Augmented Dickey Fuller
regression. 

First, we hava to determine the optimal number of lags $p_i$ for each
time-series. Several possibilities are available. They all have in
common that the maximum number of lags have to be chosen first. Then,
$p_i$ can be chosen using :

\begin{itemize}
\item the Swartz information criteria (SIC),
\item the Akaike information criteria (AIC),
\item the Hall method, which consist on removing the higher lags while
  it is not significant.
\end{itemize}

The ADF regression is run on $T-p_i-1$ observations for
each individual, so that the total number of observations is $n\times
\tilde{T}$ where $\tilde{T}=T-p_i-1$

$\bar{p}$ is the average number of lags. Call $e_{i}$ the
vector of residuals.

Estimate the variance of the $\epsilon_i$ as :

$$
\hat{\sigma}_{\epsilon_i}^2 = \frac{\sum_{t=p_i+1}^{T} e_{it}^2}{df_i}
$$



\section{Levin-Lin-Chu model}

Then, compute artificial regressions of $\Delta y_{it}$ and $y_{it-1}$
on $\Delta y_{it-L}$ and $d_{mt}$ and get the two vectors of residuals
$z_{it}$ and $v_{it}$. 

Standardize these two residuals and run the pooled regression of
$z_{it}/\hat{\sigma}_i$ on $v_{it}/\hat{\sigma}_i$ to get
$\hat{\rho}$, its standard deviation $\hat{\sigma}({\hat{\rho}})$ and
the t-statistic $t_{\hat{\rho}}=\hat{\rho}/\hat{\sigma}({\hat{\rho}})$.

Compute the long run variance of $y_i$ :

$$
\hat{\sigma}_{yi}^2 = \frac{1}{T-1}\sum_{t=2}^T \Delta y_{it}^2 + 2
\sum_{L=1}^{\bar{K}}w_{\bar{K}L}\left[\frac{1}{T-1}\sum_{t=2+L}^T
  \Delta y_{it} \Delta y_{it-L}\right]
$$

Define $\bar{s}_i$ as the ratio of the long and short term variance
and $\bar{s}$ the mean for all the individuals of the sample

$$
s_i = \frac{\hat{\sigma}_{yi}}{\hat{\sigma}_{\epsilon_i}}
$$

$$
\bar{s} = \frac{\sum_{i=1}^n s_i}{n}
$$


$$
t^*_{\rho}=\frac{t_{\rho}- n \bar{T} \bar{s} \hat{\sigma}_{\tilde{\epsilon}}^{-2}
\hat{\sigma}({\hat{\rho}}) \mu^*_{m\tilde{T}}}{\sigma^*_{m\tilde{T}}}
$$

follows a normal distribution under the null hypothesis of
stationarity. $\mu^*_{m\tilde{T}}$ and $\sigma^*_{m\tilde{T}}$ are
given in table 2 of the original paper and are also available in the
package.

\section{Im, Pesaran and Shin test}

This test does not require that $\rho$ is the same for all the
individuals. The null hypothesis is still that all the series have an
unit root, but the alternative is that some may have a unit root and
others have different values of $\rho_i <0$.

The test is based on the average of the student statistic of the
$\rho$ obtained for each individual :

$$
\bar{t}=\frac{1}{n}\sum_{i=1}^n t_{\rho i}
$$

The statistic is then :

$$
z = \frac{\sqrt{n}\left(\bar{t}- E(\bar{t})\right)}{\sqrt{V(\bar{t})}}
$$

$\mu^*_{m\tilde{T}}$ and $\sigma^*_{m\tilde{T}}$ are
given in table 2 of the original paper and are also available in the
package.
\end{document}

